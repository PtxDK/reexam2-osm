% Document class: article with font size 11pt
% ---------------
\documentclass[11pt,a4paper]{article}

\setlength{\textwidth}{165mm}
\setlength{\textheight}{240mm}
\setlength{\parindent}{0mm} % S{\aa} meget rykkes ind efter afsnit
\setlength{\parskip}{\baselineskip}
\setlength{\headheight}{0mm}
\setlength{\headsep}{0mm}
\setlength{\hoffset}{-2.5mm}
\setlength{\voffset}{0mm}
\setlength{\footskip}{15mm}
\setlength{\oddsidemargin}{0mm}
\setlength{\topmargin}{0mm}
\setlength{\evensidemargin}{0mm}

\usepackage[a4paper, hmargin={2.8cm, 2.8cm}, vmargin={2.5cm, 2.5cm}]{geometry}
\usepackage[super]{nth}
\PassOptionsToPackage{hyphens}{url}\usepackage{hyperref}
\usepackage{eso-pic} % \AddToShipoutPicture
\usepackage{float} % This will allow precise picture placement, use [H].
\usepackage{listings}
\usepackage{color}
\usepackage[titletoc,title]{appendix}

% BEGIN REDESIGN OF LSTLISTING:
\definecolor{codegreen}{rgb}{0,0.6,0}
\definecolor{codegray}{rgb}{0.5,0.5,0.5}
\definecolor{codepurple}{rgb}{0.58,0,0.82}
\definecolor{backcolour}{rgb}{0.95,0.95,0.92}

\lstdefinestyle{mystyle}{
    backgroundcolor=\color{backcolour},
    commentstyle=\color{codegreen},
    keywordstyle=\color{magenta},
    numberstyle=\tiny\color{codegray},
    stringstyle=\color{codepurple},
    basicstyle=\footnotesize,
    breakatwhitespace=false,
    breaklines=true,
    captionpos=b,
    keepspaces=true,
    numbers=left,
    numbersep=5pt,
    showspaces=false,
    showstringspaces=false,
    showtabs=false,
    tabsize=2
}

\lstset{style=mystyle}
% END REDESIGN OF LSTLISTING


% Call packages
% ---------------
\usepackage{comment} %Possible to comment larger sections
%http://get-software.net/macros/latex/contrib/comment/comment.pdf
\usepackage[T1]{fontenc} %oriented to output, that is, what fonts to use for printing characters.
\usepackage[utf8]{inputenc} %allows the user to input accented characters directly from the keyboard

%Support Windows TeXStudio
\usepackage[T1]{fontenc}
\usepackage{lmodern}

%http://mirrors.dotsrc.org/ctan/fonts/fourier-GUT/doc/latex/fourier/fourier-doc-en.pdf
\usepackage[english]{babel}														     % Danish
\usepackage[protrusion=true,expansion=true]{microtype}				                 % Better typography
%http://www.khirevich.com/latex/microtype/
\usepackage{amsmath,amsfonts,amsthm, amssymb}							 % Math packages
\usepackage[pdftex]{graphicx} %puts to pdf and graphic
%http://www.kwasan.kyoto-u.ac.jp/solarb6/usinggraphicx.pdf
\usepackage{xcolor,colortbl}
%http://mirrors.dotsrc.org/ctan/macros/latex/contrib/xcolor/xcolor.pdf
%http://texdoc.net/texmf-dist/doc/latex/colortbl/colortbl.pdf
\usepackage{tikz} %documentation http://www.ctan.org/pkg/pgf
\usepackage{parskip} %http://www.ctan.org/pkg/parskip
%http://tex.stackexchange.com/questions/51722/how-to-properly-code-a-tex-file-or-at-least-avoid-badness-10000
%Never use \\ but instead press "enter" twice. See second website for more info

% MATH -------------------------------------------------------------------
\newcommand{\Real}{\mathbb R}
\newcommand{\Complex}{\mathbb C}
\newcommand{\Field}{\mathbb F}
\newcommand{\RPlus}{[0,\infty)}
%
\newcommand{\norm}[1]{\left\Vert#1\right\Vert}
\newcommand{\essnorm}[1]{\norm{#1}_{\text{\rm\normalshape ess}}}
\newcommand{\abs}[1]{\left\vert#1\right\vert}
\newcommand{\set}[1]{\left\{#1\right\}}
\newcommand{\seq}[1]{\left<#1\right>}
\newcommand{\eps}{\varepsilon}
\newcommand{\To}{\longrightarrow}
\newcommand{\RE}{\operatorname{Re}}
\newcommand{\IM}{\operatorname{Im}}
\newcommand{\Poly}{{\cal{P}}(E)}
\newcommand{\EssD}{{\cal{D}}}
% THEOREMS ----------------------------------------------------------------
\theoremstyle{plain}
\newtheorem{thm}{Theorem}[section]
\newtheorem{cor}[thm]{Corollary}
\newtheorem{lem}[thm]{Lemma}
\newtheorem{prop}[thm]{Proposition}
%
\theoremstyle{definition}
\newtheorem{defn}{Definition}[section]
%
\theoremstyle{remark}
\newtheorem{rem}{Remark}[section]
%
\numberwithin{equation}{section}
\renewcommand{\theequation}{\thesection.\arabic{equation}}


\author{
  \Large{
    Brandt, Patrick Krøll - bwx155} \\
   \\
   %\Large{ }
}
\title{
  \huge{OSM 2016 \\}
  \Large{Operating Systems and Multiprogramming \\}
  \vspace{3cm}
  \Large{Take-Home Examination}
}

\begin{document}

\AddToShipoutPicture*{\put(0,0){\includegraphics*[viewport=0 0 700 600]{include/natbio-farve}}}
\AddToShipoutPicture*{\put(0,602){\includegraphics*[viewport=0 600 700 1600]{include/natbio-farve}}}

\AddToShipoutPicture*{\put(0,0){\includegraphics*{include/nat-en}}}

\clearpage\maketitle
\thispagestyle{empty}
\clearpage\newpage
\thispagestyle{plain}

%\tableofcontents
%\pagebreak


%<<--------------------------------------------------------------->>
\subsection*{Theoretical 1: Merge Semaphores}

An explanation was not asked for, therefore only a short one has been provided.

\begin{lstlisting}[caption={Pseudo Code},label={lst:merge-sem}]
typedef struct merge_sem_t {
    int value;
    int thread_check;
    pthread_cond_t cond;
    pthread_mutex_t lock;
} merge_sem_t;

// Only one thread can call this
void merge_sem_init(merge_sem_t *s, int value) {
    s->value = value;
    Cond_init(&s->cond);
    Mutex_init(&s->lock);
    value = 0;
    thread_check = 0;
}

void merge_sem_P1(merge_sem_t *s) {
    Mutex_lock(&s->lock);
    while (s->value <= 0 && thread_check = 0)
        Cond_wait(&s->cond, &s->lock);
        s->value--;
        Mutex_unlock(&s->lock);
    thread_check = 1;
}

void merge_sem_P2(merge_sem_t *s) {
    Mutex_lock(&s->lock);
    while (s->value <= 0 && thread_check = 1)
        Cond_wait(&s->cond, &s->lock);
        s->value--;
        Mutex_unlock(&s->lock);
    thread_check = 0;
}
\end{lstlisting}

The notation of Chapter 31 in OSTEP has been followed after best ability, [\ref{lst:merge-sem}] shows nearly the exact same as seen in OSTEP\footnote{http://pages.cs.wisc.edu/~remzi/OSTEP/threads-sema.pdf} Figure 31.16 [Version 0.91] which almost supports the requirements for this task. merge\_sem\_t, merge\_sem\_init, merge\_sem\_P1 and merge\_sem\_P2 where only a slight modification has been made by adding thread\_check, which purpose is to ensure the constant switch between the two threads.


\subsection*{Theoretical 2: Simulating MLFQ}

\begin{lstlisting}[caption={MLFQ Pseudo Code},label={lst:mlfq}]

\end{lstlisting}


\subsection*{Theoretical 3: Simulating Buddy Allocation}

This can be sufficiently solved by using the following logic.

\begin{itemize}
\item Loop through all elements in fileblocks
\item When block of big enough size is found and is also free
\item Check if block is big enough for splitting
\item Find block in fileBlocks and replace with one block, and add another block just after that block(or in practice, change the pointer from the recently changed block to point at the new block)
\end{itemize}

\begin{lstlisting}[caption={Buddy Allocation},label={lst:mlfq}]

\end{lstlisting}


\subsection*{Practical 1: Thread-Safe malloc and free}

\subsubsection*{Progression}

Firstly I read through different files to get a hold of what was available, i chose which files to read using \textbf{grep -r searchstring}, it then became clear to me that a mutex lock was already available, so I simply used the given lock by first initializing it with \textbf{heap\_init()} and then use it in malloc and free. Many different files where read through though the mainly important one has proven to be proc/syscall.c and of course userland/lib.c.

\subsubsection*{Testing}

One test has been made which shows that the malloc allocates space according to the standard in KUDOS, and does so in while also switching between threads. I have concluded that my editings work by printing the memory address in test mem\_fre.c.

\subsubsection*{Changes}

To do a short write-up, the files which has been modified are
\begin{itemize}
    \item userland/lib.c
    \item userland/Makefile
    \item userland/mem\_fre.c
\end{itemize}

\subsection*{Practical 2: Read-Write Locks}

Using the same approach as above, I did not manage to understand the implementation early enough to complete this task. A version of this is still submitted where I have added some of the things asked for in this task.


\subsection*{Appendix}



%\newpage
%\bibliography{mybib}
%\bibliographystyle{ieeetr}
\end{document}
